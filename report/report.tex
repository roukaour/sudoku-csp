\documentclass[11pt]{article}

\def\mainroot{ell_1}

\usepackage[top=1in, bottom=1in, left=1in, right=1in]{geometry}

\usepackage[compact]{titlesec}

\usepackage{xcolor}
\definecolor{pykeyword}{HTML}{000088}
\definecolor{pyidentifier}{HTML}{000000}
\definecolor{pystring}{HTML}{008800}
\definecolor{pynumber}{HTML}{FF4000}
\definecolor{pycomment}{HTML}{880000}

\usepackage{listings}
\lstset{language=Python,
	basicstyle=\ttfamily\small,
	tabsize=4,
	showstringspaces=false,
	keywordstyle=\color{pykeyword},
	identifierstyle=\color{pyidentifier},
	stringstyle=\color{pystring},
	numberstyle=\color{pynumber},
	commentstyle=\color{pycomment}
}

\usepackage{algorithm}
\usepackage{algpseudocode}

\usepackage{graphicx}

\title{CSE 537 Assignment 3 Report: Sudoku}

\author{
Remy Oukaour \\
	{\small SBU ID: 107122849}\\
	{\small \texttt{remy.oukaour@gmail.com}}
\and
Jian Yang \\
	{\small SBU ID: 110168771}\\
	{\small \texttt{jian.yang.1@stonybrook.edu}}
}

\date{Sunday, October 25, 2015}

\raggedbottom

\begin{document}

\maketitle

\section{Introduction}

This report describes our submission for assignment 3 in the CSE 537 course on
artificial intelligence. Assignment 3 requires us to implement some different
algorithms to solve Sudoku games: variations on backtracking (simple, MRV,
MRV with forwarding checking, and MRV with constraint propagation), and
min-conflicts. In this report, we discuss the implementation details and
performance of our solutions.

\section{Implementation}

\subsection{Data structures}

We implemented one class named $gameNode$ in $gameNode.py$ to contain the state of a
Sudoku board in the search tree used by our algorithms.

The $gameNode$ data structure supports operations such as setting a value, rolling back
a value, and checking if the board is solved. (The $solved$ method for performing that
check also increments the node's $num_checks$ counter.) It also supports the
constraint-related operations for different algorithms.

Two important methods of the $gameNode$ class are $get\_neighbors$ and $get\_valid\_moves$:

\lstset{language=Python}
\begin{lstlisting}[frame=single]
def get_neighbors(self, pos):
	i, j = pos
	bi = i // self.M * self.M
	bj = j // self.K * self.K
	neighbors = set()
	for k in xrange(self.N):
		neighbors.add((i, k))
		neighbors.add((k, j))
		neighbors.add((bi + k // self.K, bj + k % self.K))
	neighbors.remove(pos)
	return list(neighbors)
\end{lstlisting}

\lstset{language=Python}
\begin{lstlisting}[frame=single]
def get_valid_moves(self, pos):
	if self[pos].value():
		return []
	invalid_moves = {self[n].value() for n in self.get_neighbors(pos)}
	return [m for m in self.get_values() if m not in invalid_moves]
\end{lstlisting}

\subsection{Backtracking}

For backtracking, the algorithm chooses one arbitrary unassigned position and uses
the edge constraints to get valid values. It traverses all these valid values, and
using a backtracking depth-first search, it tries to find a solution if one exists.

The implementation of simple backtracking search in $csp.py$ is:

\lstset{language=Python}
\begin{lstlisting}[frame=single]
def backtracking(filename):
	node = gameNode.gameNode()
	if not node.load_game(filename):
		return ("Error: Fail Load", 0)
	if backtracking_helper(node):
		return node.solution()
	return ("Error: No Solution", node.num_checks)

def backtracking_helper(node):
	if node.solved():
		return True
	unassigned = node.get_unassigned_positions()
	if not unassigned:
		return False
	pos = unassigned[0]
	for move in node.get_valid_moves(pos):
		domain = node[pos].domain
		node[pos] = move
		if backtracking_helper(node):
			return True
		node[pos] = domain
	return False
\end{lstlisting}

\subsection {Backtracking + MRV}

We added the $count\_contraints$ method to the $gameNode$ class:

\lstset{language=Python}
\begin{lstlisting}[frame=single]
def count_constraints(self, pos, value=None):
	if value is None:
		value = self[pos].value()
	return len([n for n in self.get_neighbors(pos)
		if not self[n].value() and value in self.get_valid_moves(n)])
\end{lstlisting}

The implementation of backtracking + MRV in $csp.py$ is:

\lstset{language=Python}
\begin{lstlisting}[frame=single]
def backtrackingMRV(filename):
	node = gameNode.gameNode()
	if not node.load_game(filename):
		return ("Error: Fail Load", 0)
	if backtrackingMRV_helper(node):
		return node.solution()
	return ("Error: No Solution", node.num_checks)

def backtrackingMRV_helper(node):
	# Based on Figure 6.5 from page 215 of the textbook
	if node.solved():
		return True
	unassigned = node.get_unassigned_positions()
	if not unassigned:
		return False
	mrv_pos = min(unassigned, key=lambda p: len(node.get_valid_moves(p)))
	lcv_moves = sorted(node.get_valid_moves(mrv_pos),
		key=lambda m: node.count_constraints(mrv_pos, m))
	for move in lcv_moves:
		domain = node[mrv_pos].domain
		node[mrv_pos] = move
		if backtrackingMRV_helper(node):
			return True
		node[mrv_pos] = domain
	return False
\end{lstlisting}

\subsection{Backtracking + MRV + FC}

We added the $forward\_checking$ method to the $gameNode$ class:

\lstset{language=Python}
\begin{lstlisting}[frame=single]
def forward_checking(self, pos):
	# Use AC-3 with limited queue
	queue = [(n, pos) for n in self.get_neighbors(pos)
		if not self[n].value()]
	return self._ac3(queue)

def _ac3(self, queue):
	# Based on Figure 6.3 from page 209 of the textbook
	while queue:
		pos1, pos2 = queue.pop()
		if self[pos2].value() in self[pos1].domain:
			self[pos1].domain -= self[pos2].domain
			if not self[pos1].domain:
				return False
			for pos3 in set(self.get_neighbors(pos1)) - {pos2}:
				queue.append((pos3, pos1))
	return True
\end{lstlisting}

The implementation of backtracking + MRV + forward checking in $csp.py$ is:

\lstset{language=Python}
\begin{lstlisting}[frame=single]
def backtrackingMRVfwd(filename):
	node = gameNode.gameNode()
	if not node.load_game(filename):
		return ("Error: Fail Load", 0)
	if backtrackingMRVfwd_helper(node):
		return node.solution()
	return ("Error: No Solution", node.num_checks)

def backtrackingMRVfwd_helper(node):
	# Based on Figure 6.5 from page 215 of the textbook
	if node.solved():
		return True
	unassigned = node.get_unassigned_positions()
	if not unassigned:
		return False
	mrv_pos = min(unassigned, key=lambda p: len(node.get_valid_moves(p)))
	lcv_moves = sorted(node.get_valid_moves(mrv_pos),
		key=lambda m: node.count_constraints(mrv_pos, m))
	for move in lcv_moves:
		backup_board = copy.deepcopy(node.board)
		node[mrv_pos] = move
		if (node.forward_checking(mrv_pos) and
			backtrackingMRVfwd_helper(node)):
			return True
		node.board = backup_board
	return False
\end{lstlisting}


\subsection{Backtracking + MRV + CP}

We added the $propagate\_constraints$ method to the $gameNode$ class:

\lstset{language=Python}
\begin{lstlisting}[frame=single]
def propagate_constraints(self):
	# Use AC-3 with all constrained pairs
	queue = []
	for pos1 in self.get_positions():
		for pos2 in self.get_neighbors(pos1):
			if (pos2, pos1) not in queue:
				queue.append((pos1, pos2))
	return self._ac3(queue)

\end{lstlisting}

The implementation of backtracking + MRV + constraint propagation in $csp.py$ is:

\lstset{language=Python}
\begin{lstlisting}[frame=single]
def backtrackingMRVcp(filename):
	node = gameNode.gameNode()
	if not node.load_game(filename):
		return ("Error: Fail Load", 0)
	if backtrackingMRVcp_helper(node):
		return node.solution()
	return ("Error: No Solution", node.num_checks)

def backtrackingMRVcp_helper(node):
	# Based on Figure 6.5 from page 215 of the textbook
	if node.solved():
		return True
	unassigned = node.get_unassigned_positions()
	if not unassigned:
		return False
	mrv_pos = min(unassigned, key=lambda p: len(node.get_valid_moves(p)))
	lcv_moves = sorted(node.get_valid_moves(mrv_pos),
		key=lambda m: node.count_constraints(mrv_pos, m))
	for move in lcv_moves:
		backup_board = copy.deepcopy(node.board)
		node[mrv_pos] = move
		if node.propagate_constraints() and backtrackingMRVcp_helper(node):
			return True
		node.board = backup_board
	return False
\end{lstlisting}

\subsection{Min-conflicts}

We added the $count\_conflicts$ method to the $gameNode$ class:

\lstset{language=Python}
\begin{lstlisting}[frame=single]
def count_conflicts(self, pos, value=None):
	if value is None:
		value = self[pos].value()
	return len([n for n in self.get_neighbors(pos)
		if self[n].value() == value])
\end{lstlisting}

The implementation of min-conflicts in $csp.py$ is:

\lstset{language=Python}
\begin{lstlisting}[frame=single]
def minConflict(filename):
	node = gameNode.gameNode()
	if not node.load_game(filename):
		return ("Error: Fail Load", 0)
	if minConflict_helper(node):
		return node.solution()
	return ("Error: No Solution", node.num_checks)

def minConflict_helper(node, max_steps=10000):
	# Based on Figure 6.8 from page 221 of the textbook
	# initial complete assignment
	# greedy minimal-conflict values for each variable
	for pos in node.get_positions():
		if not node[pos].given:
			node[pos] = min(node.get_values(),
				key=lambda m: node.count_conflicts(pos, m))
	for _ in xrange(max_steps):
		if node.solved():
			return True
		con_pos = random.choice(node.get_conflicted_positions())
		lcv_move = min(node.get_values(),
			key=lambda m: node.count_conflicts(con_pos, m))
		node[con_pos] = lcv_move
	return False
\end{lstlisting}

\section{Results}

\subsection{Test cases}

\subsubsection{Empty Input}

As a given input in the assignment, we use an empty board to test our algorithms.
The result is shown in table \ref{tbl_bench}.

\subsubsection{Manual Cases}

We added several manual test cases (including the ones provided by Caleb on
Piazza) to compare the results.

\subsection{Benchmarks}

\begin{table}[h!]
\centering
\begin{tabular}{| c | r | r | r | r | r |}
\hline
& BT & BT+MRV & BT+MRV+FC & BT+MRV+CP & Min-conflicts \\
\hline
Empty ($9 \times 9$) & 392 & 82 & 49 & 82 & $>$10K \\
\hline
Empty($12 \times 12$) & 7426 & 145 & 95 & 145 & 40 \\
\hline
board1 & 210 & 77 & 74 & 75 & $>$10K \\
\hline
board2 & 46019 & 2482 & 2458 & 2451 & $>$10K \\
\hline
board3 & N/A & 248 & 244 & 239 & $>$10K \\
\hline
board4 & N/A & N/A & N/A & N/A & N/A \\
\hline
board5 & N/A & N/A & N/A & N/A & N/A \\
\hline
\end{tabular}
\caption{Expanded node counts for all algorithms in different test cases}
\label{tbl_bench}
\end{table}

\section{Conclusion}

\subsection{Optimizations in backtracking}

In the test result, we could see that with different extra optimization methods
to choose the next backtracking node worked much better than the simple backtracking
with arbitrary node sequence. The methods really chose ``better'' nodes to expand in
the progress as we expected and we learnt from the lectures.

\subsection{Hanging}

The min-conflicts algorithm sometimes does not solve a board, even with a high
iteration limit. We suspect that this is because it uses random choices to solve
the board, and sometimes those choices are not useful ones.

\subsection{Best algorithm}

We recommend using backtracking search with the MRV heuristic and forward checking.
This algorithm made the fewest consistency checks in our testing. MRV + constraint
propagation performed roughly the same as MRV by itself, but was much slower.
(This may be due to an inefficient implementation of constraint propagation.)
Min-heuristics sometimes made even fewer consistency checks than BT+MRV+FC, but
other times it would fail to solve the board.

\end{document}
