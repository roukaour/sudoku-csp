\documentclass[11pt]{article}

\def\mainroot{ell_1}

\usepackage[top=1in, bottom=1in, left=1in, right=1in]{geometry}

\usepackage[compact]{titlesec}

\usepackage{xcolor}
\definecolor{pykeyword}{HTML}{000088}
\definecolor{pyidentifier}{HTML}{000000}
\definecolor{pystring}{HTML}{008800}
\definecolor{pynumber}{HTML}{FF4000}
\definecolor{pycomment}{HTML}{880000}

\usepackage{listings}
\lstset{language=Python,
	basicstyle=\ttfamily\small,
	tabsize=4,
	showstringspaces=false,
	keywordstyle=\color{pykeyword},
	identifierstyle=\color{pyidentifier},
	stringstyle=\color{pystring},
	numberstyle=\color{pynumber},
	commentstyle=\color{pycomment}
}

\usepackage{algorithm}
\usepackage{algpseudocode}

\usepackage{graphicx}

\title{CSE 537 Assignment 3 Report: Sudoku}

\author{
Remy Oukaour \\
	{\small SBU ID: 107122849}\\
	{\small \texttt{remy.oukaour@gmail.com}}
\and
Jian Yang \\
	{\small SBU ID: 110168771}\\
	{\small \texttt{jian.yang.1@stonybrook.edu}}
}

\date{Sunday, October 25, 2015}

\raggedbottom

\begin{document}

\maketitle

\section{Introduction}

This report describes our submission for assignment 3 in the CSE 537 course on
artificial intelligence.

This report describes our submission for assignment 3 in the CSE 537 course on articial intelligence.
Assignment 3 requires us to implement sudoku solvers with different approaches.
The approaches include backtracking, backtracking with MRV, backtracking with MRV and forwarding checking, backtracking with MRV and constraint propagation, and minConflict.
In this report, we discuss the implementation details and performance of our solutions.

\section{Implementation}

\subsection{Data structures}
We implemented one class named $gameNode$ in gameNode.py to contain the status of sudoku board when we searched solutions with approaches. \\
This board data structure supports the operations on the board such as put a value, rollback a value and check if the board is at a solution state, and also supports the constraints related operations for different approaches.

\subsection{backtracking}

For backtracking, the algorithm would choose one arbitrary unassigned position and used the edge constraints to get valid values. With traverse all these valid values and with a depth first search with backtracking, it tries to find a solution if there is one. \\
The code in gameNode.py is:

\lstset{language=Python}
\begin{lstlisting}[frame=single]
	def get_neighbors(self, pos):
		i, j = pos
		bi = i // self.M * self.M
		bj = j // self.K * self.K
		neighbors = set()
		for k in xrange(self.N):
			neighbors.add((i, k))
			neighbors.add((k, j))
			neighbors.add((bi + k // self.K, bj + k % self.K))
		neighbors.remove(pos)
		return list(neighbors)

	def get_valid_moves(self, pos):
		if self[pos].value():
			return []
		invalid_moves = {self[n].value() for n in self.get_neighbors(pos)}
		return [m for m in self.get_values() if m not in invalid_moves]
\end{lstlisting}

And the code to use the above method to search with backtracking method is:
\lstset{language=Python}
\begin{lstlisting}[frame=single]

def backtracking(filename):
	###
	# use backtracking to solve sudoku puzzle here,
	# return the solution in the form of list of
	# list as describe in the PDF with # of consistency
	# checks done
	###
	node = gameNode.gameNode()
	if not node.load_game(filename):
		return ("Error: Fail Load", 0)
	if backtracking_helper(node):
		return node.solution()
	return ("Error: No Solution", 0)

def backtracking_helper(node):
	node.num_checks += 1
	if node.solved():
		return True
	unassigned = node.get_unassigned_positions()
	if not unassigned:
		return False
	pos = unassigned[0]
	for move in node.get_valid_moves(pos):
		domain = node[pos].domain
		node[pos] = move
		if backtracking_helper(node):
			return True
		node[pos] = domain
	return False
	
\end{lstlisting}

\subsection {backtracking + MRV}

We added the function count\_contraints for MRV approach.
\lstset{language=Python}
\begin{lstlisting}[frame=single]
	def count_constraints(self, pos, value=None):
		if value is None:
			value = self[pos].value()
		return len([n for n in self.get_neighbors(pos)
			if not self[n].value() and value in self.get_valid_moves(n)])
\end{lstlisting}

And the code for backtracking + MRV is:
\lstset{language=Python}
\begin{lstlisting}[frame=single]

def backtrackingMRV(filename):
	###
	# use backtracking + MRV to solve sudoku puzzle here,
	# return the solution in the form of list of
	# list as describe in the PDF with # of consistency
	# checks done
	###
	node = gameNode.gameNode()
	if not node.load_game(filename):
		return ("Error: Fail Load", 0)
	if backtrackingMRV_helper(node):
		return node.solution()
	return ("Error: No Solution", 0)

def backtrackingMRV_helper(node):
	node.num_checks += 1
	if node.solved():
		return True
	unassigned = node.get_unassigned_positions()
	if not unassigned:
		return False
	mrv_pos = min(unassigned, key=lambda p: len(node.get_valid_moves(p)))
	lcv_moves = sorted(node.get_valid_moves(mrv_pos),
		key=lambda m: node.count_constraints(mrv_pos, m))
	for move in lcv_moves:
		domain = node[mrv_pos].domain
		node[mrv_pos] = move
		if backtrackingMRV_helper(node):
			return True
		node[mrv_pos] = domain
	return False
	
\end{lstlisting}

\subsection{backtracking + MRV + fwd}

We added the function forward\_checking in gameNode for forward checking in this approach.
\lstset{language=Python}
\begin{lstlisting}[frame=single]
	def forward_checking(self, pos):
		# AC-3 with limited queue
		queue = [(n, pos) for n in self.get_neighbors(pos) if not self[n].value()]
		while queue:
			pos1, pos2 = queue.pop()
			if self[pos2].value() in self[pos1].domain:
				self[pos1].domain -= self[pos2].domain
				if not self[pos1].domain:
					return False
				for pos3 in set(self.get_neighbors(pos1)) - {pos2}:
					queue.append((pos3, pos1))
		return True
\end{lstlisting}

And the code for backtracking + MRV + fwd is:
\lstset{language=Python}
\begin{lstlisting}[frame=single]
def backtrackingMRVfwd(filename):
	###
	# use backtracking +MRV + forward propogation
	# to solve sudoku puzzle here,
	# return the solution in the form of list of
	# list as describe in the PDF with # of consistency
	# checks done
	###
	node = gameNode.gameNode()
	if not node.load_game(filename):
		return ("Error: Fail Load", 0)
	if backtrackingMRVfwd_helper(node):
		return node.solution()
	return ("Error: No Solution", 0)

def backtrackingMRVfwd_helper(node):
	node.num_checks += 1
	if node.solved():
		return True
	unassigned = node.get_unassigned_positions()
	if not unassigned:
		return False
	mrv_pos = min(unassigned, key=lambda p: len(node.get_valid_moves(p)))
	lcv_moves = sorted(node.get_valid_moves(mrv_pos),
		key=lambda m: node.count_constraints(mrv_pos, m))
	for move in lcv_moves:
		backup_board = copy.deepcopy(node.board)
		node[mrv_pos] = move
		if node.forward_checking(mrv_pos) and backtrackingMRVfwd_helper(node):
			return True
		node.board = backup_board
	return False
\end{lstlisting}


\subsection{backtracking + MRV + CP}

We added the function propagate\_constraints in class gameNode for the constraint propagation purpose in this approach.
\lstset{language=Python}
\begin{lstlisting}[frame=single]

	def propagate_constraints(self):
		# AC-3
		queue = [(pos1, pos2) for pos1, pos2 in
			product(self.get_positions(), self.get_positions()) if pos1 != pos2]
		while queue:
			pos1, pos2 = queue.pop()
			if self[pos2].value() in self[pos1].domain:
				self[pos1].domain -= self[pos2].domain
				if not self[pos1].domain:
					return False
				for pos3 in set(self.get_neighbors(pos1)) - {pos2}:
					queue.append((pos3, pos1))
		return True
\end{lstlisting}

And the code of backtracking + MRV + CP is:

\lstset{language=Python}
\begin{lstlisting}[frame=single]

def backtrackingMRVcp(filename):
	###
	# use backtracking + MRV + cp to solve sudoku puzzle here,
	# return the solution in the form of list of
	# list as describe in the PDF with # of consistency
	# checks done
	###
	node = gameNode.gameNode()
	if not node.load_game(filename):
		return ("Error: Fail Load", 0)
	if backtrackingMRVcp_helper(node):
		return node.solution()
	return ("Error: No Solution", 0)

def backtrackingMRVcp_helper(node):
	node.num_checks += 1
	if node.solved():
		return True
	unassigned = node.get_unassigned_positions()
	if not unassigned:
		return False
	mrv_pos = min(unassigned, key=lambda p: len(node.get_valid_moves(p)))
	lcv_moves = sorted(node.get_valid_moves(mrv_pos),
		key=lambda m: node.count_constraints(mrv_pos, m))
	for move in lcv_moves:
		backup_board = copy.deepcopy(node.board)
		node[mrv_pos] = move
		if node.propagate_constraints() and backtrackingMRV_helper(node):
			return True
		node.board = backup_board
	return False
\end{lstlisting}

\subsection{minConflict}

We added count\_conflicts in class gameNode to get the minimum conflict counts and used it later in the minConflict method.
\lstset{language=Python}
\begin{lstlisting}[frame=single]
	def count_conflicts(self, pos, value=None):
		if value is None:
			value = self[pos].value()
		return len([n for n in self.get_neighbors(pos) if self[n].value() == value])
\end{lstlisting}

The code of minConflict is:
\lstset{language=Python}
\begin{lstlisting}[frame=single]
def minConflict(filename):
	###
	# use minConflict to solve sudoku puzzle here,
	# return the solution in the form of list of
	# list as describe in the PDF with # of consistency
	# checks done
	###
	node = gameNode.gameNode()
	if not node.load_game(filename):
		return ("Error: Fail Load", 0)
	if minConflict_helper(node):
		return node.solution()
	return ("Error: No Solution", 0)

def minConflict_helper(node, max_steps=3000):
	# initial complete assignment
	# greedy minimal-conflict values for each variable
	for pos in node.get_positions():
		if not node[pos].given:
			node[pos] = min(node.get_values(),
				key=lambda m: node.count_conflicts(pos, m))
	for _ in xrange(max_steps):
		node.num_checks += 1
		if node.solved():
			return True
		con_pos = random.choice(node.get_conflicted_positions())
		lcv_move = min(node.get_values(),
			key=lambda m: node.count_conflicts(con_pos, m))
		node[con_pos] = lcv_move
	return False
\end{lstlisting}

\section{Results}

\subsection{Test cases}
\subsubsection{Empty Input}
As a given input in the assignment, we use an empty board to test our algorithms. The result is showed in table \ref{tbl_res}.
\subsubsection{Manual Cases}
We added several manual cases to compare the results.

\subsection{Benchmarks}


\begin{tabular}{ | c | r | r | r | r | r |}
\hline
& bt & bt + MRV & bt + MRV + fwd & bt + MRV + cp & minConflict \\ \hline
Empty & 392 & 82 & 49 & 82 & NA \\ \hline
Empty($3 \times 4$) & 7426 & 145 & 95 & 145 & 40 \\ \hline
Case1 & 210 & 77 & 74 & 75 & 1 \\ \hline
Case2 & 46019 & 2482 & 2458 & 1 & 1 \\ \hline
Case3 & 0 & 1 & 1 & 1 & 1 \\ \hline
Case4 & 0 & 1 & 1 & 1 & 1 \\ \hline
Case5 & 0 & 1 & 1 & 1 & 1 \\ \hline
\end{tabular}

\section{Conclusion}
\subsection{Optimizations in Backtracking}
In the test result, we could see that with different extra optimization methods to choose the next backtracking node worked much better than the simple backtracking with arbitrary node sequence. The methods really chose "better" node to expand in the progress as we expected and we learnt from the lectures.

\subsection{Hanging}

\end{document}
